% Use: pdflatex SQLInterview.tex
% git add SQLInterview.tex
% git commit -m "something"
% git push origin master

\documentclass[a4paper,11pt]{article}
\usepackage{soul}
\usepackage{url}
\begin{document}
\title{Microsoft SQL Server Developer Interview Questions}
\author{\url{postgresql2016@gmail.com}}
\date{Feburary 1, 2016}
\maketitle
The following interview questoins are all from the REAL interviews(including both telephone and on-site interviews). Hope this document is helpful during your job hunting process. Enjoy!
\section{MS SQL Server and T-SQL}
Q: \ul{If the SQL performance is bad, tell me how you can fix it}? \newline
A: This is a typical and difficult question. Usually the SQL in the production database meets some performance trouble. The quickest way is to check the statistics of the related tables(You can find the table list from the SQL code), and update the statistics very quickly. If the statistics are updated, in most case, MS SQL Server will find the best execution plan by itself, and the trouble is fixed. However, if you do not see any performance improvement after you update the statistics of all tables, then you need to capture the execution plan from the memory(The SQL is still running/hung there, so you can easily extract the real execution plan from the memory) and compare it with the old execution plan. MS SQL Server store the old execution plan in the database. If the execution plan is changed, then you can use the old but good execution plan to apply on the current SQL. As the last resort, you should review the SQL code and rewrite it, or add some hints in the code. This is the basic idea about how to trouble shoot the SQL performance issue in the production database.\newline \newline
\noindent 
Q: What are the tools you use to find, identify and fix the bad performance SQL query? \newline \newline
\noindent 
Q: Could you share with us some of your exeperience about performance tunning SQL code and SSIS package? \newline \newline
\noindent 
Q: \ul{What is the difference between DELETE and TRUNCATE commands in T-SQL}? \newline 
A: Please see below:
\begin{description}
  \item[$\bullet$] DELETE statement removes rows one at a time and records an entry in the transaction log for each deleted row. TRUNCATE operation removes the data by deallocating the data pages used to store the table’s data and only the page deallocations are recorded in the transaction log. (The deallocation of data pages means that your data rows still actually exist in the data pages, but the extents have been marked as empty for reuse.)?
  \item[$\bullet$] TRUNCATE  is a faster operation to perform over DELETE
  \item[$\bullet$] TRUNCATE will reset any identity columns to the default seed value in its identity columns. DELETE will not do this.
  \item[$\bullet$] TRUNCATE TABLE removes all rows from a table, but the table structure and its columns, constraints, indexes and so on remain. 
  \item[$\bullet$] Cannot use TRUNCATE TABLE on a table referenced by a FOREIGN KEY constraint; Participate in an indexed view, published by using transactional replication or merge replication. 
  \item[$\bullet$] TRUNCATE TABLE cannot activate a trigger. 
\end{description}
\noindent 
\newline
Q: What is CTE? What is CTE can do but normal T-SQL cannot do?\newline \newline
\noindent 
Q: \ul{I can DROP PROCEDURE and CREATE PROCEDURE, and I can also ALTER PROCEDURE. What is the difference}? \newline 
A: ALTER PROCEDURE retains any permissions that have been established for this stored procedure. It keeps the same object ID with system objects and allows the dependenceies to be kept. If you perform a DROP and then use a CREATE, you have almost the same effect as using an ALTER PROC statement with one rather big difference — if you DROP and CREATE, you will need to entirely reestablish your permissions on who can and can’t use this stored procedure.\newline\newline
\noindent 
Q: How do you know acceptable transaction size? \newline \newline
\noindent 
Q: How to add a constraint to let your table accept data more than 5 years old? \newline \newline
\noindent 
Q: Please select all tables name containing column called 'somename'. \newline \newline
\noindent 
Q: Please list the table with more than 2 dependencies \newline \newline
\noindent 
Q: Tell me how to use MERGE statement. \newline \newline
\noindent 
Q: Do you know the CASE type? \newline \newline
\noindent 
Q: What is CURSOR? \newline \newline
\noindent 
Q: How do you check if your data is modified successfully in production? \newline \newline
\noindent 
Q: What is the difference between INDEX SEEK and INDEX SCAN? \newline \newline
\noindent 
Q: When you design a table, what is the way to ensure that each row is unique, what way do you accomplish this? \newline \newline
\noindent 
Q: How do you make DATETIME data type displaying as DATE without TIME? \newline \newline
\noindent 
Q: \ul{What is the size of the database you haved worked on}? \newline
A: It depends on the type of the database. We have three database types: OLTP(Online Transaction Processing) database, ODS(Operational Data Store) database, EDW(Enterprise Data Warehouse) database. OLTP: 500 GB - 3 TB; ODS: 10 TB - 60 TB; EDW: 5 TB - 30 TB. OLTP is much smaller, and the biggest one is ODS because it contains all the history raw data for many years.\newline \newline
\noindent 
Q: What is ODS and what is EDW. In ODS, what kind of data are you storing? How large is the ODS store in your current job?\newline \newline
\noindent 
Q: What is the challenge when getting the data from your ODS to the Data Warehouse?\newline \newline
\noindent 
Q: What is the typical data load window from your ODS to the EDW? What is the data loading volumne?\newline \newline
\noindent 
Q: Are you part of the Data Warehouse design? What is the challenge during the data warehouse design?\newline \newline
\noindent 
Q: When you assign a value to a variable, what is the difference between 'SET' and 'SELECT'? \newline \newline
\noindent 
Q: \ul{When you have INNER JOIN with condition 'ON 1=1', what kind of expected result will you have}? \newline 
A: It will get the result of CROSS JOIN! 'SELECT xxxx FROM tableA INNER JOIN tableB ON (1=1)' is the same thing as 'SELECT xxxx FROM tableA CROSS JOIN tableB'. \newline \newline
\noindent 
Q: What is CROSS JOIN, INNER JOIN, OUTER JOIN, FULL JOIN? \newline \newline
\noindent 
Q: As a SQL developer, could you tell me why I need or do not need the cluster index? Just generally speaking.\newline \newline
\noindent 
Q: What is cluster index, and what is non-cluster index?\newline \newline
\noindent 
Q: \ul{What is the difference between PRIMARY KEY and UNIQUE constraint}?\newline
A: Primary key doesn't allow NULLs, but unique key allows one NULL only. Primary key creates a clustered index on the column by default, unique index creates a nonclustered index by default. Of course you can create nonclustered primary key and clustered unique index.\newline\newline
\noindent 
Q: What is the advantage and disadvantage of using GUID as the unique key compared with Integer type? \newline \newline
\noindent 
Q: What is the consideration when you design a new query to extract the data from a very large table to ensure the good performance? \newline \newline
\noindent 
Q: \ul{What is the syntax order of FROM, HAVING, SELECT, ORDER BY, GROUP BY, WHERE, INTO in a T-SQL SELECT statement?} \newline
A: SELECT $\rightarrow$ INTO $\rightarrow$ FROM $\rightarrow$ WHERE $\rightarrow$ GROUP BY $\rightarrow$ HAVING $\rightarrow$ ORDER BY. \newline \newline
\noindent 
Q: What is the difference between WHERE clause and HAVING clause? \newline \newline
\noindent 
Q: \ul{What is the execution order of FROM, HAVING, SELECT, ORDER BY, GROUP BY, WHERE in a T-SQL SELECT statement?} \newline
A: FROM $\rightarrow$ WHERE $\rightarrow$ GROUP BY $\rightarrow$ HAVING $\rightarrow$ SELECT $\rightarrow$ ORDER BY. \newline \newline
\noindent 
Q: Giving you the freedom to design something from the beginning, what normal form do you typically use for the development? \newline \newline
\noindent 
Q: What does the optimizer do to handle the INNER/LEFT JOIN condition and WHERE clause? \newline \newline
\noindent 
Q: What is the index view with SCHEMABINDING option in MS SQL Server? \newline \newline
\noindent 
Q: \ul{Please list three different ways to import data into MS SQL Server.} \newline
A: Bulk Insert, BCP, SQL Server Import/Export Wizard, and you can write the SSIS package to do that. You can also use some script tools such as Perl, Python to read the data text file and generate the INSERT sql script to import the data.\newline\newline
\noindent 
Q: What is the difference between table variable and temp table? \newline \newline
\noindent 
Q: How to share the output of one SQL query with another SQL session? \newline \newline
\noindent 
Q: What is the set option to suppress the row count output of one SQL session? \newline \newline
\noindent 
Q: What is the difference between row compression and page compression? \newline \newline
\noindent 
Q: What is the VIEW and when should it be used? \newline \newline
\noindent 
Q: What is IDENTITY column? \newline \newline
\noindent 
Q: What is SYNONYM in Microsoft SQL Server? \newline \newline
\noindent 
Q: You have a requirement to extract every LASTNAME from table A that is not in table B. How to do that? \newline \newline
\noindent 
Q: What is Data Warehouse? Can you explain what is the difference between Kimball and Inmon data warehouse? \newline \newline
\noindent 
Q: What is the PIVOT statement in T-SQL? \newline \newline
\noindent 
Q: How about the OUTPUT clause in T-SQL? \newline \newline
\noindent 
Q: What is STAR schema? What is SNOWFLAKE schema? \newline \newline
\noindent 
Q: How do you handle the errors in the stored procedure? \newline \newline
\noindent 
Q: Tell me something about TRY...CATCH in T-SQL, and how do you use it? \newline \newline
\noindent 
Q: What is partition table? \newline \newline
\noindent 
Q: What is the difference between data mart and data store? \newline \newline

\section{SSIS}
Q: Tell me how to load Type 2 dimension table in SSIS? \newline \newline
\noindent 
Q: How would you say managing changing dimension in your data warehouse? Slowly changing dimesion is only one of the managing changing dimension.\newline \newline
\noindent 
Q: What is the difference between Type 1 and Type 2 SCD? \newline \newline
\noindent 
Q: What is the difference between Script Component and Script Task? \newline \newline
\noindent 
Q: What is advantage and disadvantage of C\# and T-SQL? \newline \newline
\noindent 
Q: The SSIS package is failed in production, how do you make it continue running? \newline \newline
\noindent 
Q: I need to incremental deploy my package(A project with package A and B under Project Deployment mode). A needs to be fixed, B is good. How to deploy A only? \newline \newline
\noindent 
Q: Give me an example about how to use C\# in SSIS package. \newline \newline
\noindent 
Q: There is a SSIS package that loads data from one flat file into a SQL Server table. It failed in the middle night, and you are called to do the troubleshooting. What are the steps you want to do? \newline \newline
\noindent 
Q: How to get rid of the duplicate records from the millions of records from the data that you want to import into SQL server to avoid unique constraint violation? \newline \newline
\noindent 
Q: What is the difference between synchronous transformation and asynchronous transformation in SSIS? \newline \newline
\noindent 
Q: When you are working on SSIS packages, what type of configuration have you done to them, how can you perform the configurations? \newline \newline
\noindent 
Q: How can you deploy the SSIS packages into production environment? \newline \newline
\noindent 
Q: In SSIS, what is the difference between the variables and the parameters? \newline \newline
\noindent 
Q: What is sequence container in SSIS. Do you know how many type of containers in SSIS package? \newline \newline
\noindent 
Q: We have a store procedure that is called by our ETL package. It is working slowly. How could you find what problem is, and what kind of solution you will use to fix it?\newline \newline
\noindent 
Q: For the control flow and data flow in SSIS package, which one can be run parallelly, which one can only be running sequencely? \newline \newline
\noindent 
Q: \ul{We will get one file each day from the vendor, but someday we get many files from this vendor. How to handle mulitple files in your SSIS package}? \newline 
A: I think we should design the package using the For-Each-Loop Container. If the files are in one folder or directory, the For-Each-Loop container will automatically find these files, no matter one file or many files there. (This is a good hands-on lab, so please do it by youself).\newline\newline
\noindent 
Q: Are you familiar with the Data Pipeline Object in SSIS? \newline \newline
\noindent 
Q: What else ETL tools do you beside SSIS? \newline \newline
\noindent 
Q: What kind of tools either than MS BI stack, in package design, data schema design, code do you use in your team? \newline \newline

\section{SSRS}
Q: List all the ways you know to deploy a report.\newline \newline
\noindent 
Q: What's the difference between DRILL-DOWN and DRILL-THROUGH? \newline \newline


\section{SSAS}
-- This is the SSAS question.

\section{Development and Others}
Q: \ul{What are the project management tools you are using in your team}?\newline
A: Please google JIRA, RALLY, DAILY SCRUM MEETING, USER STORY and prepare these terms. We also use Subversion before as the source code version control tool, but now we turn to Git and Github. Please also google Subvresion, Git, Github by yourself.\newline \newline
\noindent 
Q: Tell me about something for Microsoft Azure. \newline \newline
\noindent 
Q: What is the relationship between class and object in C\#? \newline \newline
\noindent 
Q: Do you have any experience with Oracle database? \newline \newline

\section{Behavior Questions}
Q: \ul{Tell me a little bit of your background, what you are doing in your current position}? \newline
A: This is the first question the interviewer wants to ask you. They want to get some overall impression from your statements. You should prepare your introductoin carefully before your interview. You can listen to many good introductions from our audios. DO PLEASE prepare it in advance by yourself and write the statements down in the papaer and palce the papaer in front of you before the interview.\newline \newline
\noindent 
Q: In your daily jobs, what is the most important task you do? \newline \newline
\noindent 
Q: What is the biggest challenge in your current project? \newline \newline
\noindent 
Q: How do you keep up with new technology? \newline \newline
\noindent 
Q: From the career path perspective, what are you looking to do and how soon are you going to do it? \newline \newline
\noindent 
Q: \ul{For the team member working with you and the manager, how would they describe you}? \newline 
A: high technical skills plus easy going personality. \newline\newline
\noindent 
Q: What is the most challenge things, and what are you struggling with? \newline \newline
\noindent 
Q: What is the reason why you want to leave current position and look for new opportunity? \newline \newline
\noindent 
Q: \ul{Do you have any questions you want to ask from us}?\newline
A: This is the last question the interviewer wants to ask you. This is the last opportunity that you can bring good impression to the interviewer. If you ask some high quality questions, it will be very good for your interview.
\begin{description}
  \item[$\bullet$ question 1:] What is your expectation from me if I can get this offer?
  \item[$\bullet$ question 2:] What is biggest challenge in your team? What is the biggest pressure your team is facing?
  \item[$\bullet$ question 3:] How many developers in your team? Do you have DBA in your team?
  \item[$\bullet$ question 4:] What is the size of your biggest database?
\end{description}

\noindent 
--- THE END --- 

\end{document}
